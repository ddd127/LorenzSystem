% "Станет проще"

\documentclass[12pt]{article}

% report, book

%  Русский язык

\usepackage[T2A]{fontenc}
\usepackage[utf8]{inputenc}
\usepackage[english,russian]{babel}

\usepackage{geometry}
\geometry{a4paper}

\usepackage{amsmath,amsfonts,amssymb,amsthm,mathtools} 


\usepackage{wasysym}

\newtheorem*{Lemma}{Лемма}
\theoremstyle{definition}
\newtheorem*{Remark}{Замечание}
\newtheorem*{Remark1}{Замечание 1}
\newtheorem*{Remark2}{Замечание 2}
\newtheorem*{Sign}{Обозначение}
\newtheorem*{Claim}{Утверждение}
\newtheorem*{Feat}{Свойства}
\newtheorem*{Cor}{Следствие}
\newtheorem*{Theorem}{Теорема}
\theoremstyle{definition}
\newtheorem*{Examples}{Примеры}
\newtheorem*{Example}{Пример}
\newtheorem*{known}{Теорема}
\newtheorem*{Proof}{Доказательство}
\newtheorem*{SpecCase}{Частный случай}
\theoremstyle{definition}
\newtheorem*{Def}{Определение}

\usepackage{geometry} 
\geometry{a4paper,top=2cm,bottom=3cm,left=2cm}



\begin{document} % начало документа

\newpage

\section{Система Лоренца}

Аттрактор Лоренца ― \textbf{компактное инвариантное множество} L в трехмерном фазовом пространстве гладкого потока

\section{Конвекция в замкнутой петле}

Трубка, замкнутая в кольцо, наполнена почти несжимаемой жидкостью. Она подогревается снизу и охлаждается сверху, и при достаточно сильном нагреве возможно возникновение конвекционного течения.

-- рисуночек --

Заметим, что раз жидкость почти несжимаемая, то ее скорость во всех точках трубки постоянна, она не зависит от $\phi$. Обозначим скорость за $X$.

Температура жидкости в трубке будет зависить от угла $\phi$, который отсчитывается от направленного вниз радиуса до точки против часовой стрелки. $T = T(\phi) -$ периодическая функция (период равен $2\pi$) $\Rightarrow$ ее можно разложить в ряд Фурье.

При разложении основную амплитуду задают первые слагаемые, поэтому рассмотрим только первую гармонику. Уравнение будет иметь вид $T = T_0 \cdot (1+Ysin\phi + Zcos\phi)$. Исследуем коэффициенты при $sin\phi$ и $cos\phi$.

Отклонение температуры от радиуса, направленного к нагревателю зависит от $cos \phi$, значит $Z$ характеризует отклонение температуры от среднего значения в точке нагрева (нижней точки трубки). Отклонение от крайней правой точки (на 3 часа, то есть $\phi = \frac{\pi}{2}$) зависит от $sin\phi$, и $Y$ его характеризует.

Рассмотрим, что вызывает изменение скорости жидкости. Во-первых, действует сила Архимеда, она пропорциональна $Y$, так как в $\phi=\frac{\pi}{2}$ она направлена вверх. Во-вторых, действует сила вязкости, которая пропорциональна скорости, следовательно, пропорциональна $X$. Тогда

\begin{equation}
	\dot{X} = cY-\beta X
\end{equation}

Пусть течение происходит с постоянной скоростью, то есть угол изменяется на $Xt$. Тогда

\begin{equation}
	T=f(\phi-Xt)=T_0(1+Ysin(\phi-Xt)+Zcos(\phi-Xt))
\end{equation}

Пусть $\phi'=\phi-Xt$, тогда

\begin{equation}
	\dot{T}=T_0(Y(-X)cos\phi'-Z(-X)sin\phi')=T_0(ZXsin\phi'-YXcos\phi')
\end{equation}

Вспомним, что за отклонение температуры от нижней точки отвечал $cos\phi \Longrightarrow$ перенос температуры потоком жидкости для $Z$ учитывается членом $-YX$. Аналогично, для $Y$ будет $ZX$.

В точке $Z$ происходит постоянный подогрев, поэтому в уравнении будем его учитывать, добавив константу $A$. 

В системе стремится установиться термодинамическое равновесие, следовательно, надо учитывать в уравнениях для $Y$ и $Z$ релаксацию, это будут члены $-DY$ и $-DZ$ соотвественно, где $D -$ постоянная величина.

Таким образом

$$\begin{cases}
	\dot{X} = cY-\beta X \\
	\dot{Y} = XZ-DY \\
	\dot{Z} = A-XY-DZ
\end{cases}$$

Заменим переменные: $X=Dx, Y=\frac{\beta Dy}{c}, Z=-\frac{\beta Dz}{c}$, тогда

$$\begin{cases}	
	\dot{x} = \sigma (y-x) \\
	\dot{y} = x(r-z)-y \\
	\dot{z} = xy-bz
\end{cases}$$

где $\sigma = \frac{\beta}{D}, r=\frac{cA}{\beta D^2}, b=1$.


\end{document} % конец документа